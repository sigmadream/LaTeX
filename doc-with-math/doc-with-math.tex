\documentclass[a4paper,amsmath]{oblivoir}

\usepackage{fapapersize}
\usefapapersize{*,*,1in,*,1in,*}
\usepackage[default=false,exception={vmatrix}]{ob-mathleading}
\usepackage{mathtools}
\usepackage{listings}
\usepackage{xcolor}

\newcommand\pkg[1]{\textsf{#1}}

\usepackage{tcolorbox}
\tcbuselibrary{listings,breakable}

\usepackage{lipsum}  
\usepackage{blindtext}

\begin{document}

\title{\LaTeX을 활용한 간단한 문서 작성 예제}
\author{한상곤(sangkon@pusan.ac.kr)}
\date{\today, v1.0.0}
\maketitle

\begin{abstract}
    \lipsum[2-3]
\end{abstract}

\tableofcontents*

\section{서론}

\blindtext

본문의 문단과 별행 수식 사이의 간격과 같은 것은 이 패키지가 제어하지 아니하며 ``여러 줄 수식''에 대해서만
효과가 있다. 이 패키지를 사용하지 않았을 때 \pkg{oblivoir} 문서를 작성한 결과와 이 패키지를 사용한 결과를 비교하여 보아라. 
아래는 \pkg{ob-mathleading}를 사용하지 않았을 때이다.
\[
    \begin{bmatrix} a & b & c \\ p & q & f \\ g & j & i \end{bmatrix},
    \begin{bmatrix} 1 & 0 \\ 0 & 1\end{bmatrix},
    \begin{bmatrix} 1 & 0 & \ast\! \\ 0 & 1 & \ast\! \end{bmatrix},
    \begin{bmatrix} 1 & 0 \\ 0 & 1 \\ 0 & 0 \end{bmatrix},
    \begin{pmatrix} 1 & 0 & 0 \\ 0 & 1 & 0 \\ 0 & 0 & 1 \end{pmatrix}
\]
아래는  \pkg{ob-mathleading}를 사용했을 때의 조판이다.
\begin{obMathLeading}
    \[
        \begin{bmatrix} a & b & c \\ p & q & f \\ g & j & i \end{bmatrix},
        \begin{bmatrix} 1 & 0 \\ 0 & 1\end{bmatrix},
        \begin{bmatrix} 1 & 0 & \ast\! \\ 0 & 1 & \ast\! \end{bmatrix},
        \begin{bmatrix} 1 & 0 \\ 0 & 1 \\ 0 & 0 \end{bmatrix},
        \begin{pmatrix} 1 & 0 & 0 \\ 0 & 1 & 0 \\ 0 & 0 & 1 \end{pmatrix}
    \]
\end{obMathLeading}

\section{본론}

\subsection{본론1}

문서의 preamble에 패키지의 사용을 선언한다.

\begin{tcblisting}{listing only}
    \usepackage{ob-mathleading}
\end{tcblisting}

``적절한'' 행간격을 얻기 위해서는 이렇게 선언하는 것으로 충분하다. 패키지의 수식 행간격 기본값은 $1$이다.

\subsection{본론2}

일관성있는 문서를 작성하기 위해서는 문서 전체에 걸쳐서 동일한 수식 행간격을 유지하는 것이 바람직하다. 
그러나 부득이하게 특정 부분이나 특정 수식에 대해서만 행간을 늘리거나 줄여야 할 필요가 있을 수 있다.
이 패키지는 이럴 경우에 대응하기 위하여 하나의 명령(선언)과 하나의 환경을 제공한다.

\begin{tcblisting}{listing only}
    \obmathleading{<value>}
\end{tcblisting}

\verb|<value>|에는 stretch 값을 \verb|1.2|, \verb|2|와 같이 준다. 이 선언 이후에는 수식 행간격이 변경되며 이 변경은 지역적(\emph{local})이므로
현재의 범위(scope) 내에서 유효하다.

\begin{tcblisting}{listing only}
    \begin{obMathLeading}[<value>]
        ...
    \end{obMathLeading}
\end{tcblisting}

\verb|obMathLeading| 환경은 이 환경 안에 오는 수식에만 영향을 끼친다. 옵션 인자가 주어지지 않으면
기본값인 $1$이 사용되며 옵션 인자가 주어지면 그것을 stretch로 사용하여 행간격을 조절한다.
환경 안에는 \pkg{amsmath}의 여러 줄 수식 환경이 올 수 있다.

\subsection{본론3}

\subsubsection{본론3-세부1 \texttt{mathleading}}

문서 전체에 걸쳐 수식 행간격을 임의로 조절하고자 한다면
\begin{tcblisting}{listing only}
    \usepackage[mathleading=<value>]{ob-mathleading}
\end{tcblisting}
\noindent 와 같이 지정한다. \verb|<value>|에는 stretch 값이 온다. 패키지의 기본값은 \verb|1|이다.

\subsubsection{본론3-세부2  \texttt{exception}}

이 패키지가 영향을 미치는 \pkg{amsmath}의 수식 환경의 리스트는 다음과 같다.

\begin{quote}
    \begin{ttfamily}
        array, matrix, pmatrix, bmatrix, Bmatrix, vmatrix, Vmatrix,
        cases, align, aligned, alignat, alignedat, gather, gathered,
        split, multline, xalignat, xxalignat
    \end{ttfamily}
\end{quote}

이 가운데 행간 조절 적용을 배제하고자 하는 환경의 이름을 쉼표로 분리하고 중괄호로 묶어서 열거할 수 있다.

\begin{tcblisting}{listing only}
    \usepackage[exception={cases,vmatrix}]{ob-mathleading}
\end{tcblisting}

다만 이 경우에 \texttt{align, alignat, xalignat, xxalignat}는 모두 동일한
것으로 취급되며 \texttt{align}만을 대표 이름으로 다룬다. \texttt{align}과 \texttt{aligned}는
서로 다른 환경이므로 별도로 취급된다.

다음 보기는 \verb|exception={vmatrix}| 옵션을 준 상태의 예시이다.

\begin{tcblisting}{listing above text}
    \begin{obMathLeading}
        \[
            \begin{bmatrix} a & b & c \\ p & q & f \\ g & j & i \end{bmatrix},
            \begin{bmatrix} 1 & 0 \\ 0 & 1\end{bmatrix},
            \begin{bmatrix} 1 & 0 & \ast\! \\ 0 & 1 & \ast\! \end{bmatrix},
            \begin{bmatrix} 1 & 0 \\ 0 & 1 \\ 0 & 0 \end{bmatrix},
            \begin{vmatrix} 1 & 0 & 0 \\ 0 & 1 & 0 \\ 0 & 0 & 1 \end{vmatrix}
        \]
    \end{obMathLeading}
\end{tcblisting}

\subsubsection{본론3-세부3  \texttt{default}}

일반적인 상황은 아니겠지만 수식 행간격을 조절하는 이 패키지의 기능을 사용하지 않으면서 단지 명령과 환경, \verb|\obmathleading|이나
\verb|obMathLeading|만을 활용하고자 한다면
\begin{tcblisting}{listing only}
    \usepackage[default=false]{ob-mathleading}
\end{tcblisting}
이와 같이 \verb|[default=false]|를 옵션으로 줄 수 있다. 패키지의 기본값은 \verb|true|이다.

\subsubsection{본론3-세부4  \texttt{noallowdisplaybreaks}}

이 패키지는 \verb|\allowdisplaybreaks|를 실행해준다. 이 기능을 억제하고자 한다면
패키지 옵션으로 \verb|[noallowdisplaybreaks]|를 선언한다. 마지막의 \verb|s|를 빠뜨리지 않도록
유의하라.

\section{논의 및 토의}

이 패키지는 (이름에 나타난 바와 같이) \pkg{oblivoir}를 위하여 작성되기는 하였지만 \pkg{oblivoir} 클래스에
의존하지 않는다. 그러므로 다른 클래스의 문서에서도 동작할 것이다.

\pkg{amsmath}는 필수적이며 \pkg{etoolbox}를 로드한다.

\section{결론}

테스트 문서와 예제 소스를 제공하신 ktug 게시판의 Progress 님, 패키지 제작의 동기를 제공하신 조인성 교수께 감사드린다.

\begin{lstlisting}[language=Python, caption={Python example}]
    import numpy as np
        
    def incmatrix(genl1,genl2):
        m = len(genl1)
        n = len(genl2)
        M = None #to become the incidence matrix
        VT = np.zeros((n*m,1), int)  #dummy variable
        
        #compute the bitwise xor matrix
        M1 = bitxormatrix(genl1)
        M2 = np.triu(bitxormatrix(genl2),1) 
    
        for i in range(m-1):
            for j in range(i+1, m):
                [r,c] = np.where(M2 == M1[i,j])
                for k in range(len(r)):
                    VT[(i)*n + r[k]] = 1;
                    VT[(i)*n + c[k]] = 1;
                    VT[(j)*n + r[k]] = 1;
                    VT[(j)*n + c[k]] = 1;
                    
                    if M is None:
                        M = np.copy(VT)
                    else:
                        M = np.concatenate((M, VT), 1)
                    
                    VT = np.zeros((n*m,1), int)
        
        return M
    \end{lstlisting}
\end{document}